\documentclass[]{beamer}
\mode<presentation>
{
% use attribute "dark" for dark theme
% use attribute "contrast" for maximum text contrast
\usetheme[]{wlansi}
\usefonttheme[onlymath]{serif}
\setbeamercovered{transparent}
}  

\usepackage{hyperref}
\usepackage[utf8]{inputenc}
\usepackage{mathptmx}
\usepackage{xmpmulti}
\usepackage[T1]{fontenc}
\usepackage[utf8]{inputenc}
\usepackage{cmap}
\usepackage{ifthen}
\usepackage{type1ec}
\usepackage{concrete}
\DeclareFontFamily{T1}{ccr}{}
\DeclareFontShape{T1}{ccr}{m}{n}{<5><6><7><8><9><10>gen*eorm<10->eorm10}{}
\DeclareFontShape{T1}{ccr}{m}{sl}{<5><6><7><8><9><10>gen*eosl<10->eosl10}{}
\DeclareFontShape{T1}{ccr}{m}{it}{<->eoti10}{}
\DeclareFontShape{T1}{ccr}{m}{sc}{<->eocc10}{}
\DeclareFontShape{T1}{ccr}{bx}{n}{<->ssub*cmr/bx/n}{}
\DeclareFontShape{T1}{ccr}{bx}{sl}{<->ssub*cmr/bx/sl}{}
\DeclareFontShape{T1}{ccr}{bx}{it}{<->ssub*cmr/bx/it}{}
\DeclareFontShape{T1}{ccr}{sbc}{n}{<->ssubf*ecssdc10}{}
\usepackage[T1]{fontenc}
\renewcommand{\sfdefault}{\rmdefault}
\renewcommand{\familydefault}{\rmdefault}
\renewcommand{\bfdefault}{m}
\newcommand{\nodewatcher}{\emph{nodewatcher }}
\newcommand{\wlanslovenija}{\emph{wlan slovenija }}

\newcommand{\ifstrempty}[3]{%
\def\reallyempty{}%
\def\ifarg{#1}%
\ifx\ifarg\reallyempty%
{#2}
\else
{#3}
\fi%
}

\newcommand{\wrapframe}[1]{
\begin{frame}
 #1
\end{frame}
}

\newenvironment{changemargin}[2]{%
\begin{list}{}{%
\setlength{\topsep}{0pt}%
\setlength{\leftmargin}{#1}%
\setlength{\rightmargin}{#2}%
\setlength{\listparindent}{\parindent}%
\setlength{\itemindent}{\parindent}%
\setlength{\parsep}{\parskip}%
}%
\item[]}{\end{list}}

\newcommand{\simpleslide}[3]{\wrapframe{
\ifstrempty{#1}{\vbox{ \ } 

}{
{\footnotesize #1}
} \vfill

\begin{center}
#2
\end{center}
\vfill
\ifstrempty{#3}{ \ }{
\begin{flushright}
{\footnotesize #3}
\end{flushright}
}
}}

\newcommand{\simpleslideimage}[2]{\wrapframe{
\begin{center}
\includegraphics[width=0.8\paperwidth,height=0.6\paperheight,keepaspectratio]{#1}
\\ {\footnotesize #2}
\end{center}}
}

\newcommand{\fullslideimage}[1]{
\begin{frame}[plain]
\begin{changemargin}{-1cm}{-1cm}
\begin{center}
\includegraphics[width=\paperwidth,height=\paperheight,keepaspectratio]{#1}
\end{center}
\end{changemargin}
\end{frame}
}

\newcommand{\rightfooter}[1]{\vfill\hfill #1}

\title{Kaj je \wlanslovenija omrežje?}

%\subtitle{}

%\author{}

%\institute{}

\date[]{1.0
}

\begin{document}

% makes a title slide
\maketitle

\pgfdeclarelayer{background}
\pgfsetlayers{background,main}

\simpleslide{}{\wlanslovenija -- odprto brezžično omrežje Slovenije \\ http://wlan-si.net}{http://grow.wlan-si.net \\ http://dev.wlan-si.net}

\simpleslide{Zamisli si \dots}{\dots prost dostop do Interneta pri tebi doma.}{}

\simpleslide{Zamisli si \dots}{\dots prost dostop do Interneta kjerkoli po tvojem kraju.}{}

\simpleslide{Zamisli si \dots}{\dots prost dostop do Interneta po celotni Sloveniji.}{}

\simpleslide{Da, to je mogoče.}{\huge To je \emph{wlan slovenija}.}{} 

\simpleslide{}{Pobuda za izgradnjo odprtega brezžičnega \\
omrežja po Sloveniji, temelječega na sodelovanju, \\
izmenjavi znanja in izkušenj ter skupnosti.}{}

\simpleslide{Internet ni nekaj snovnega, neka surovina, \\ ki bi jo nekje nekdo ustvarjal in potem distribuiral.}{Internet je možnost komunikacije, \\ možnost povezljivosti z drugimi.}{In to povezljivost lahko vsakdo širi, \\ izboljšuje, gradi infrastrukturo \dots}

\simpleslide{}{Povezljivost v Internet, kjer sicer ni mogoča.}{Vzeti povezljivost v svoje roke.}

\simpleslide{}{Imaš povezljivost v Internet? Podeli jo!}{Izmenjajmo povezljivost med seboj.}

\simpleslide{}{Uporabljaš \wlanslovenija omrežje? Razširi ga naprej!}{Izmenjajmo povezljivost med seboj.}

\simpleslide{Odprto brezžično omrežje.}{Plod sodelovanja posameznikov, skupin in organizacij.}{Brezžično mesh omrežje.}

\simpleslide{\wlanslovenija je:}{Preprost, varen in organiziran način deljenja \\ in širjenja Internetne povezljivosti.}{Za vsakogar.}

\simpleslide{\wlanslovenija je:}{Samostojno, organsko omrežje.}{Za vsakogar.}

\simpleslide{\wlanslovenija je:}{Odprto okolje za tehnični razvoj.}{Za vsakogar.}

\simpleslide{\wlanslovenija je:}{Razvoj tehnologij, družbe \dots}{Za vsakogar.}

\simpleslide{Deljenje in širjenje Internetne povezljivosti.}{\begin{enumerate}
\item Uporabiš preprost brezžični router.
\item Povežeš ga s svojim obstoječim Internetnim priključkom.
\item Postaviš na okensko polico, balkon, streho \dots
\end{enumerate}}{Povezljivost v Internet povsod.}

\simpleslide{Tako ustvarjamo:}{Oprto/raznoliko/organsko omrežje na nivoju države.}{}

\simpleslide{2006 -- Predstavitev mesh omrežja FunkFeuer v Kiberpipi. \\ Navdušenci pograbijo idejo.}
{2009 -- \emph{wlan ljubljana} -- odprto brezžično omrežje Ljubljane \\ Poenostavitev in avtomatizacija postavitev točk. \\ Omrežje naj lahko gradi vsakdo. \\ 28 točk, 13000 ne-unikatnih uporabnikov.}
{2012 -- \wlanslovenija -- odprto brezžično omrežje Slovenije \\
150+ točk, 180000 ne-unikatnih uporabnikov.}

\simpleslideimage{images/growth.png}{Rast omrežja (v letu 2011).}

\simpleslide{Internetna povezljivost v Sloveniji (2010).}{
V 60 \% gospodinjstvih širokopasovni Internet (xDSL, kabelski). \\
Optika (FTTH) v večjih mestih. \\
76 \% populacije uporablja Internet. \\
500 \% rast dostopnosti do širokopasovnega Interneta '05--'10. \\
Slaba dostopnost v ruralnih območjih.}{}

\simpleslide{}{Neizkoriščena širokopasovna infrastruktura v mestih.\\ $\downarrow$ \\
Deljenje povezljivosti za omogočanje \\ dostopa na javnih površinah.}{}

\simpleslide{}{Pomanjkanje širokopasovne infrastrukture v ruralnih območjih.\\ $\downarrow$ \\
Deljenje povezljivosti za osnovni dostop.}{}

\simpleslide{Neuspeh.}{Občina v Halozah uvede spletni portal za občane.}{Velik del gospodinjstev nima možnosti širokopasovnega dostopa do Internet.}

\simpleslide{}{Prebivalci vzpostavijo svojo povezljivost v Internet.}{Uspeh!}

\fullslideimage{images/haloze.png}

\simpleslide{Cilji.}{Preprosto in učinkovito deljenje in \\ širjenje Internete povezljivosti.}{}

\simpleslide{Cilji.}{Razvoj preprostih, učinkovitih in odprtokodnih rešitev.}{Primer: \emph{nodewatcher}.}

\simpleslide{Cilji.}{Deljenje in ustvarjanje znanja.}{Primer: delavnice.}

\simpleslide{Cilji.}{Spodbujanje novih vsebin in storitev v brezžičnih omrežjih.}{Primer: odprto družabno omrežje znotraj omrežja, PiplMesh.}

\simpleslide{Da, lahko podeliš svojo obstoječo Internetno povezljivost!}{\begin{enumerate}
	\item Kaj, če kdo zasede mojo celotno povezavo?
	\item Kaj, če kdo zlorabi mojo povezavo?
	\item Kaj, če kdo vdre v moj računalnik?
\end{enumerate}}{Sevanje.}

\simpleslide{Gradimo omrežje.}{Uporabljamo vse načine povezovanja točk med seboj.}{WiFi, ethernet, tunele, optiko \dots}

\fullslideimage{clipart/shema-podolgovata.png}

% TODO: Shema iz vidika enega routerja, več v ticketu #1048

\simpleslide{Uporaba prostih kapacitet na obstoječih infrastrukturah.}{S tem se vzpostavlja lastna infrastruktura \\ (točke se povezujejo med seboj).}{Gradnja lastne hrbtenice nad vsem tem.}

\simpleslide{}{Adaptivno, dinamično, organsko.}{}

\simpleslide{Rezultati.}{Samostojno omrežje.}{Vedno bolj.}

\simpleslide{Rezultati.}{Neomejena povezljivost.}{Ampak ne prepustnost.}

\simpleslide{Rezultati.}{Brezplačno.}{Za uporabo.}

\simpleslideimage{images/dilbert.jpg}{}

\simpleslide{}{Zastavljeno okoli skupnosti.}{Vsakdo lahko sodeluje.}

\simpleslide{Skupnost.}{Se učimo. Sodelujemo. Izmenjujemo znanja, izkušnje \dots}{Vsi smo enkrat začeli.}

\simpleslide{Skupnost.}{Omrežje tvorijo sodelujoči.}{}

\simpleslide{Skupno omrežje.}{Vsebine. \\ Storitve. \\ Uporabniki.}{Brezžično. Mobilno. Nov medij za komunikacije.}

\simpleslide{Odprta koda.}{Linux.}{OpenWrt. \\ http://www.openwrt.org/}

\simpleslide{}{Kako vse skupaj izgleda?}{}

\fullslideimage{images/slovenija.png}

\fullslideimage{images/maribor.png}

\fullslideimage{images/ljubljana.png}

\fullslideimage{images/solar-1.jpg}

\fullslideimage{images/solar-2.jpg}

\fullslideimage{images/neubergerjeva.jpg}

\fullslideimage{images/glinska.jpg}

\fullslideimage{images/rozmanova-1.jpg}

\fullslideimage{images/rozmanova-2.jpg}

\fullslideimage{images/urban.jpg}

\fullslideimage{images/linkslatinaurban06.jpg}

\simpleslide{}{Prekomejno povezovanje.}{}

\fullslideimage{images/PohorjeTestiranje.png}

\simpleslideimage{images/awmn.png}{AWMN, 2400 točk (Atene, Grčija)}

\simpleslideimage{images/guifi.png}{guifi.net, 17000 točk (Španija)}

\end{document}
