\documentclass[]{beamer}
\mode<presentation>
{
% use attribute "dark" for dark theme
% use attribute "contrast" for maximum text contrast
\usetheme[]{wlansi}
\usefonttheme[onlymath]{serif}
\setbeamercovered{transparent}
}  

\usepackage{hyperref}
\usepackage[utf8]{inputenc}
\usepackage{mathptmx}
\usepackage{xmpmulti}
\usepackage[T1]{fontenc}
\usepackage[utf8]{inputenc}
\usepackage{cmap}
\usepackage{ifthen}
\usepackage{type1ec}
\usepackage{concrete}
\DeclareFontFamily{T1}{ccr}{}
\DeclareFontShape{T1}{ccr}{m}{n}{<5><6><7><8><9><10>gen*eorm<10->eorm10}{}
\DeclareFontShape{T1}{ccr}{m}{sl}{<5><6><7><8><9><10>gen*eosl<10->eosl10}{}
\DeclareFontShape{T1}{ccr}{m}{it}{<->eoti10}{}
\DeclareFontShape{T1}{ccr}{m}{sc}{<->eocc10}{}
\DeclareFontShape{T1}{ccr}{bx}{n}{<->ssub*cmr/bx/n}{}
\DeclareFontShape{T1}{ccr}{bx}{sl}{<->ssub*cmr/bx/sl}{}
\DeclareFontShape{T1}{ccr}{bx}{it}{<->ssub*cmr/bx/it}{}
\DeclareFontShape{T1}{ccr}{sbc}{n}{<->ssubf*ecssdc10}{}
\usepackage[T1]{fontenc}
\renewcommand{\sfdefault}{\rmdefault}
\renewcommand{\familydefault}{\rmdefault}
\renewcommand{\bfdefault}{m}
\newcommand{\nodewatcher}{\emph{nodewatcher }}
\newcommand{\wlanslovenija}{\emph{wlan slovenija }}

\newcommand{\ifstrempty}[3]{%
\def\reallyempty{}%
\def\ifarg{#1}%
\ifx\ifarg\reallyempty%
{#2}
\else
{#3}
\fi%
}

\newcommand{\wrapframe}[1]{
\begin{frame}
 #1
\end{frame}
}

\newenvironment{changemargin}[2]{%
\begin{list}{}{%
\setlength{\topsep}{0pt}%
\setlength{\leftmargin}{#1}%
\setlength{\rightmargin}{#2}%
\setlength{\listparindent}{\parindent}%
\setlength{\itemindent}{\parindent}%
\setlength{\parsep}{\parskip}%
}%
\item[]}{\end{list}}

\newcommand{\simpleslide}[3]{\wrapframe{
\ifstrempty{#1}{\vbox{ \ } 

}{
{\footnotesize #1}
} \vfill

\begin{center}
#2
\end{center}
\vfill
\ifstrempty{#3}{ \ }{
\begin{flushright}
{\footnotesize #3}
\end{flushright}
}
}}

\newcommand{\simpleslideimage}[3][width]{\wrapframe{
\begin{center}
\ifthenelse{\equal{#1}{width}}{
\includegraphics[width=\pwidth]{#2}}{
\ifthenelse{\equal{#1}{height}}{
\includegraphics[height=\pheight]{#2}}{
\includegraphics{#2}}}
\\ {\footnotesize #3}
\end{center}}
}

\newcommand{\fullslideimage}[1]{
\begin{frame}[plain]
\begin{changemargin}{-1cm}{-1cm}
\begin{center}
\includegraphics[width=\paperwidth,height=\paperheight,keepaspectratio]{#1}
\end{center}
\end{changemargin}
\end{frame}
}

\newcommand{\rightfooter}[1]{\vfill\hfill #1}

\newlength{\pheight}
\newlength{\pwidth}
\setlength{\pheight}{0.8\paperheight}
\setlength{\pwidth}{0.8\paperwidth}

\title{Kaj je \wlanslovenija omrežje?}

%\subtitle{}

%\author{}

%\institute{}

\date[] {1.1}

\begin{document}

% makes a title slide
\maketitle

\pgfdeclarelayer{background}
\pgfsetlayers{background,main}

\simpleslide{}{\emph{wlan slovenija} -- odprto brezžično omrežje Slovenije \\ http://wlan-si.net}{http://grow.wlan-si.net \\ http://dev.wlan-si.net}

\simpleslide{}{Motivacija?}{}

\simpleslide{}{Dostop do Interneta, kjer sicer ni mogoč.}{Vzeti povezljivost v svoje roke.}

\simpleslide{}{Imaš dostop do Interneta? Podeli ga!}{Izmenjajmo povezljivost med seboj.}

\simpleslide{}{Uporabljaš deljeno povezavo? Podeli jo naprej!}{Izmenjajmo povezljivost med seboj.}

\simpleslide{Odprto brezžično omrežje.}{
Plod sodelovanja posameznikov, organizacij.\\
Preprost, varen, organiziran način deljenja Internetne povezave.\\
Razvoj tehnologij, družbe \dots
}{Brezžično mesh omrežje.}

\simpleslide{\wlanslovenija je:}{
Deljenje lastne Internetne povezave.\\
Samostojno, organsko omrežje.\\
Odprto okolje za tehnični razvoj.\\
}{Za vsakogar.}

\simpleslide{Deljenje Internetne povezave.}{
Uporabiš preprost brezžični router.\\
Povežeš ga s svojim Internetnim priključkom.\\
Postaviš na okensko polico, streho \dots\\
}{Dostop do Interneta povsod.}

\simpleslide{Tako ustvarjamo:}{Oprto/raznoliko/organsko omrežje na nivoju države.}{}

\fullslideimage{clipart/shema-podolgovata.png}

\simpleslide{Da, lahko ponudiš svojo povezavo!}{\begin{enumerate}
	\item Kaj, če kdo zasede mojo celotno povezavo?
	\item Kaj, če kdo zlorabi mojo povezavo?
	\item Kaj, če kdo vdre v moj računalnik?
\end{enumerate}}{Sevanje.}

\simpleslide{Omrežje.}{Mesh omrežje.}{Brezžično mesh omrežje.}

\simpleslide{}{Uporabljamo vse načine povezovanja točk med seboj.}{WiFi, ethernet, tunele, optiko \dots}

\fullslideimage{clipart/shema-podolgovata.png}

\simpleslide{}{Adaptivno, dinamično, organsko.}{}

\simpleslide{}{Zastavljeno okoli skupnosti.}{Vsakdo lahko sodeluje.}

\simpleslide{Skupnost.}{Se učimo. Sodelujemo. Izmenjujemo znanja, izkušnje \dots}{Vsi smo enkrat začeli.}

\simpleslide{Skupnost.}{Omrežje tvorijo sodelujoči.}{}

\simpleslide{Skupno omrežje.}{Vsebine. \\ Storitve. \\ Uporabniki.}{Brezžično. Mobilno.}

\simpleslide{Odprta koda.}{Linux.}{OpenWrt. \\ http://www.openwrt.org/}

\simpleslide{}{Kako vse skupaj izgleda?}{}

\fullslideimage{clipart/omrezje.png}

\fullslideimage{clipart/preplet.png}

\simpleslideimage[height]{clipart/solar-1.jpg}{}

\simpleslideimage[height]{clipart/solar-2.jpg}{}

\simpleslideimage[height]{clipart/neubergerjeva.jpg}{}

\simpleslideimage[height]{clipart/glinska.jpg}{}

\simpleslideimage[height]{clipart/rozmanova-1.jpg}{}

\simpleslideimage[height]{clipart/rozmanova-2.jpg}{}

\simpleslideimage[height]{clipart/urban.jpg}{}

\simpleslideimage[height]{clipart/linkslatinaurban06.jpg}{}

\simpleslide{}{\wlanslovenija -- odprto brezžično omrežje Slovenije \\ http://wlan-si.net}{http://grow.wlan-si.net \\ http://dev.wlan-si.net}

\simpleslide{2006 -- Predstavitev mesh omrežja FunkFeuer v Kiberpipi. \\ Navdušenci pograbijo idejo.}
{2009 -- \emph{wlan ljubljana} -- odprto brezžično omrežje Ljubljane \\ Poenostavitev in avtomatizacija postavitev točk. \\ 28 točk, 13000 ne-unikatnih uporabnikov}
{2012 -- \wlanslovenija -- odprto brezžično omrežje Slovenije \\
150+ točk, 178000 ne-unikatnih uporabnikov}

\simpleslideimage[width]{clipart/growth.png}{Rast omrežja (v letu 2011).}

\simpleslide{Internetna povezljivost v Sloveniji.}{
Širokopasovni Internet (xDSL ali kableski) -- 60 \% gospodinjstev -- '10 \\
FTTH v večjih mestih \\
76\% populacije uporablja Internet \\
500\% rast dostopnosti do širokopasovnega Interneta '05-'10 \\
slaba dostopnost v ruralnih obnočjih}{}

\simpleslide{}{Neizkoriščena širokopasovna infrastruktura v mestih.\\ $\downarrow$ \\
Deljenje povezljivosti za omogočanje \\ dostopa na javnih površinah.}{}

\simpleslide{}{Pomanjkanje širokopasovne infrastrukture v ruralnih območjih.\\ $\downarrow$ \\
Deljenje povezljivosti za osnovni dostop.}{}

\simpleslide{Neuspeh}{
Občine v Halozah uvajajo spletne portale za občane.\\
Velik del gospodinjstev nima možnosti širokopasovnega dostopa.\\
~\\
Prebivalci vzpostavijo svoje omrežje.\\
}{Uspeh!}

\simpleslideimage[width]{clipart/haloze.png}{Brezžično omrežje v Halozah.}

\simpleslide{Cilji!}{
Preprosto in učinkovito deljenje Interneta.\\
Razvoj preprostih, učinkovitih in odprtokodnih rešitev.\\
Promocija odprtih družbenih omrežij.\\
Deljenje znanja.\\
}{}

\simpleslide{}{\emph{wlan slovenija} -- odprto brezžično omrežje Slovenije \\ http://wlan-si.net}{http://grow.wlan-si.net \\ http://dev.wlan-si.net}

\simpleslide{Omrežje}{Mesh omrežje.}{Brezžično mesh omrežje.}

\simpleslideimage[height]{clipart/slovenija.png}{}
\simpleslideimage[height]{clipart/preplet.png}{}

\simpleslide{Širjenje omrežja:}{Sodelovanje posameznikov.  \\ Deljenje lokacije. \\ Prispevanje denarja. \\ }{Prekomejna povezljivost.}
\simpleslideimage[height]{clipart/PohorjeTestiranje.png}{}
\simpleslide{Rezultati.}{Avtonomno omrežje.  \\ Neomejena povezljivost. \\Brezplačno. \\ }{Nov medij za komunikacije. }
\simpleslideimage[width]{clipart/dilbert.jpg}{}
\simpleslideimage[width]{clipart/awmn1}{AWMN Athenes}
\simpleslideimage[width]{clipart/awmn2}{AWMN Athenes}

% TODO: guifi.net

\end{document}
