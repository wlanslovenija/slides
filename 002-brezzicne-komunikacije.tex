\documentclass[]{beamer}
\mode<presentation>
{
% use attribute "dark" for dark theme
% use attribute "contrast" for maximum text contrast
\usetheme[]{wlansi}
\usefonttheme[onlymath]{serif}
\setbeamercovered{transparent}
}  

\usepackage{hyperref}
\usepackage[utf8]{inputenc}
\usepackage{mathptmx}
\usepackage{xmpmulti}
\usepackage[T1]{fontenc}
\usepackage[utf8]{inputenc}
\usepackage{cmap}
\usepackage{ifthen}
\usepackage{type1ec}
\usepackage{concrete}
\DeclareFontFamily{T1}{ccr}{}
\DeclareFontShape{T1}{ccr}{m}{n}{<5><6><7><8><9><10>gen*eorm<10->eorm10}{}
\DeclareFontShape{T1}{ccr}{m}{sl}{<5><6><7><8><9><10>gen*eosl<10->eosl10}{}
\DeclareFontShape{T1}{ccr}{m}{it}{<->eoti10}{}
\DeclareFontShape{T1}{ccr}{m}{sc}{<->eocc10}{}
\DeclareFontShape{T1}{ccr}{bx}{n}{<->ssub*cmr/bx/n}{}
\DeclareFontShape{T1}{ccr}{bx}{sl}{<->ssub*cmr/bx/sl}{}
\DeclareFontShape{T1}{ccr}{bx}{it}{<->ssub*cmr/bx/it}{}
\DeclareFontShape{T1}{ccr}{sbc}{n}{<->ssubf*ecssdc10}{}
\usepackage[T1]{fontenc}
\renewcommand{\sfdefault}{\rmdefault}
\renewcommand{\familydefault}{\rmdefault}
\renewcommand{\bfdefault}{m}
\newcommand{\nodewatcher}{\emph{nodewatcher }}
\newcommand{\wlanslovenija}{\emph{wlan slovenija }}

\newcommand{\ifstrempty}[3]{%
\def\reallyempty{}%
\def\ifarg{#1}%
\ifx\ifarg\reallyempty%
{#2}
\else
{#3}
\fi%
}

\newcommand{\wrapframe}[1]{
\begin{frame}
 #1
\end{frame}
}

\newenvironment{changemargin}[2]{%
\begin{list}{}{%
\setlength{\topsep}{0pt}%
\setlength{\leftmargin}{#1}%
\setlength{\rightmargin}{#2}%
\setlength{\listparindent}{\parindent}%
\setlength{\itemindent}{\parindent}%
\setlength{\parsep}{\parskip}%
}%
\item[]}{\end{list}}

\newcommand{\simpleslide}[3]{\wrapframe{
\ifstrempty{#1}{\vbox{ \ } 

}{
{\footnotesize #1}
} \vfill

\begin{center}
#2
\end{center}
\vfill
\ifstrempty{#3}{ \ }{
\begin{flushright}
{\footnotesize #3}
\end{flushright}
}
}}

\newcommand{\simpleslideimage}[2]{\wrapframe{
\begin{center}
\includegraphics[width=0.8\paperwidth,height=0.6\paperheight,keepaspectratio]{#1}
\\ {\footnotesize #2}
\end{center}}
}

\newcommand{\fullslideimage}[1]{
\begin{frame}[plain]
\begin{changemargin}{-1cm}{-1cm}
\begin{center}
\includegraphics[width=\paperwidth,height=\paperheight,keepaspectratio]{#1}
\end{center}
\end{changemargin}
\end{frame}
}

\newcommand{\rightfooter}[1]{\vfill\hfill #1}

\title{Osnove brezžičnih komunikacij}

%\subtitle{}

%\author{}

%\institute{}

\date[]{1.0
}

\begin{document}

% makes a title slide
\maketitle

\simpleslideimage{images/AnalogueAM.png}{Analogni amplitudno moduliran (AM) signal}

\simpleslideimage{images/DigitalAM.png}{Digitalni amplitudno moduliran (AM) signal}

\simpleslideimage{images/AMradio.png}{AM radijski sprejemnik}

\simpleslide{Izboljšave.}{Drugi načini moduliranja.}{}

\simpleslide{Izboljšave.}{
Boljša spektralna učinkovitost. \\
Boljša energijska učinkovitost. \\
Večja prepustnost. \\
Večja kompleksnost.}{}

\simpleslide{}{WiFi (802.11n) je moduliran z OFDM \\ (Orthogonal frequency-division multiplexing).}{}

\fullslideimage{images/FrequencySpectrum.png}

\simpleslide{Antena.}{Pretvarja električne signale v elektromagnetno valovanje.}{}

\simpleslide{Antena.}{Vsaka frekvenca zahteva primerne dimenzije antene.}{}

\simpleslide{Antena.}{Najpreprostejša antena -- half-wave dipol -- polvalni dipol. \\
$\lambda = \frac{c}{f}$}{}

\simpleslide{Karakteristike anten.}{Dobitek -- gain (dBi) -- usmerjenost antene. \\
SWR -- razmerje izsevane in od antene odbite moči -- kvaliteta. \\
Impedanca -- WiFi 50 $\Omega$. \\
Polarizacija -- vodoravna/horizontalna ali krožna.}{}

\simpleslide{Tipi anten.}{
Omni-directional. \\
Slot. \\
Panel. \\
Helix. \\
Yagi. \\
3D corner.}{}

\simpleslideimage{images/OmniPattern2dB.png}{Omni antena 2 dBi}

\simpleslideimage{images/OmniPattern5dB.png}{Omni antena 5 dBi}

\simpleslideimage{images/PanelPattern.png}{Panel antena}

\simpleslideimage{images/SectorPattern.png}{Sektor antena}

\simpleslideimage{images/YagiPattern.png}{Yagi antena}

\simpleslideimage{images/Polarization.png}{Polarizacija antene}

\simpleslideimage{images/LinkLoss.png}{Brezžični sistem, izgube}

\simpleslide{Komunikacija.}{Razlika/razmerje med jakostjo signala in šuma.}{}

\simpleslide{Signal.}{Kar oddaja tvoj oddajnik.}{}

\simpleslide{Šum.}{Kar oddajajo drugi oddajaniki.}{}

\simpleslide{Ali se poveča jakost signala.}{}{Ali se zmanjša šum.}

\simpleslide{Bolj povečujemo jakost signala.}{}{Bolj povečujemo šum drugim.}

\simpleslide{}{In to se seveda dogaja enako nazaj.}{}

\simpleslideimage{images/DrugaLink.png}{Izračun signala med dvema točkama}

\end{document}
