\documentclass[]{beamer}
\mode<presentation>
{
% use attribute "dark" for dark theme
% use attribute "contrast" for maximum text contrast
\usetheme[]{wlansi}
\usefonttheme[onlymath]{serif}
\setbeamercovered{transparent}
}  

\usepackage{hyperref}
\usepackage[utf8]{inputenc}
\usepackage{mathptmx}
\usepackage{xmpmulti}
\usepackage[T1]{fontenc}
\usepackage[utf8]{inputenc}
\usepackage{cmap}
\usepackage{ifthen}
\usepackage{type1ec}
\usepackage{concrete}
\DeclareFontFamily{T1}{ccr}{}
\DeclareFontShape{T1}{ccr}{m}{n}{<5><6><7><8><9><10>gen*eorm<10->eorm10}{}
\DeclareFontShape{T1}{ccr}{m}{sl}{<5><6><7><8><9><10>gen*eosl<10->eosl10}{}
\DeclareFontShape{T1}{ccr}{m}{it}{<->eoti10}{}
\DeclareFontShape{T1}{ccr}{m}{sc}{<->eocc10}{}
\DeclareFontShape{T1}{ccr}{bx}{n}{<->ssub*cmr/bx/n}{}
\DeclareFontShape{T1}{ccr}{bx}{sl}{<->ssub*cmr/bx/sl}{}
\DeclareFontShape{T1}{ccr}{bx}{it}{<->ssub*cmr/bx/it}{}
\DeclareFontShape{T1}{ccr}{sbc}{n}{<->ssubf*ecssdc10}{}
\usepackage[T1]{fontenc}
\renewcommand{\sfdefault}{\rmdefault}
\renewcommand{\familydefault}{\rmdefault}
\renewcommand{\bfdefault}{m}
\newcommand{\nodewatcher}{\emph{nodewatcher }}
\newcommand{\wlanslovenija}{\emph{wlan slovenija }}

\newcommand{\ifstrempty}[3]{%
\def\reallyempty{}%
\def\ifarg{#1}%
\ifx\ifarg\reallyempty%
{#2}
\else
{#3}
\fi%
}

\newcommand{\wrapframe}[1]{
\begin{frame}
 #1
\end{frame}
}

\newenvironment{changemargin}[2]{%
\begin{list}{}{%
\setlength{\topsep}{0pt}%
\setlength{\leftmargin}{#1}%
\setlength{\rightmargin}{#2}%
\setlength{\listparindent}{\parindent}%
\setlength{\itemindent}{\parindent}%
\setlength{\parsep}{\parskip}%
}%
\item[]}{\end{list}}

\newcommand{\simpleslide}[3]{\wrapframe{
\ifstrempty{#1}{\vbox{ \ } 

}{
{\footnotesize #1}
} \vfill

\begin{center}
#2
\end{center}
\vfill
\ifstrempty{#3}{ \ }{
\begin{flushright}
{\footnotesize #3}
\end{flushright}
}
}}

\newcommand{\simpleslideimage}[3][width]{\wrapframe{
\begin{center}
\ifthenelse{\equal{#1}{width}}{
\includegraphics[width=\pwidth]{#2}}{
\ifthenelse{\equal{#1}{height}}{
\includegraphics[height=\pheight]{#2}}{
\includegraphics{#2}}}
\\ {\footnotesize #3}
\end{center}}
}

\newcommand{\fullslideimage}[1]{
\begin{frame}[plain]
\begin{changemargin}{-1cm}{-1cm}
\begin{center}
\includegraphics[width=\paperwidth,height=\paperheight,keepaspectratio]{#1}
\end{center}
\end{changemargin}
\end{frame}
}


\newcommand{\rightfooter}[1]{\vfill\hfill #1}

\newlength{\pheight}
\newlength{\pwidth}
\setlength{\pheight}{0.6\paperheight}
\setlength{\pwidth}{0.8\paperwidth}

\title{Osnove brezžičnih komunikacij}

%\subtitle {Sežana, 23.6.2012}

%\author{Luka Čehovin}

%\institute{wlan slovenija}

\date[] {1.0}

\begin{document}

% makes a title slide
\maketitle

\simpleslideimage[width]{clipart/AnalogueAM.png}{Analogni amplitudno moduliran signal}
\simpleslideimage[width]{clipart/DigitalAM.png}{Digitalni amplitudno moduliran signal}
\simpleslideimage[width]{clipart/AMradio.png}{AM radijski sprejemnik}
\simpleslide{izboljšave}{
večina drugih načinov moduliranja:
\begin{itemize}
\item boljša spektralna učinkovitost
\item boljša energijska učinkovitost
\item večja prepustnost
\item večja kompleksnost
\end{itemize}
WiFi-802.11n je moduliran z OFDM (Orthogonal frequency-division multiplexing)
}{modulrianje}
\fullslideimage{clipart/FrequencySpectrum.png}

\simpleslide{antena}{
pretvarja električne signale v elektromagnetno valovanje
}{}

\simpleslide{antene}{
vsaka frekvenca zahteva primerne dimenzije antene

najpreprostejša antena -- half-wave dipol -- polvalni dipol

$\lambda = \frac{c}{f}$
}{}

\simpleslide{antene}{
karakteristike:
\begin{itemize}
\item dobitek -- gain (dB) -- usmerjenost antene
\item SWR -- razmerje izsevane in od antene odbite moči -- kvaliteta
\item impedanca -- WiFi $50\Omega$
\item polarizacija -- vodoravna/horizontalna ali krožna
\end{itemize}
}{}


\simpleslide{antene}{
tipi:
\begin{itemize}
\item omni-directional
\item slot
\item panel
\item helix
\item yagi
\item 3D corner
\item ...
\end{itemize}
}{}

\simpleslideimage[height]{clipart/OmniPattern2dB.png}{Omni antena 2dB}
\simpleslideimage[height]{clipart/OmniPattern5dB.png}{Omni antena 5dB}
\simpleslideimage[height]{clipart/PanelPattern.png}{Panel antena}
\simpleslideimage[height]{clipart/SectorPattern.png}{Sector antena}
\simpleslideimage[height]{clipart/YagiPattern.png}{Yagi antena}
\simpleslideimage[width]{clipart/Polarization.png}{Polarizacija antene}
\simpleslideimage[width]{clipart/LinkLoss.png}{Brezžični sistem, izgube}

\simpleslideimage[width]{clipart/DrugaLink.png}{Izračun signala med dvema točkama}

\end{document}
