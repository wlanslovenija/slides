\documentclass[]{beamer}
\mode<presentation>
{
% use attribute "dark" for dark theme
% use attribute "contrast" for maximum text contrast
\usetheme[]{wlansi}
\usefonttheme[onlymath]{serif}
\setbeamercovered{transparent}
}  

\usepackage{hyperref}
\usepackage[utf8]{inputenc}
\usepackage{mathptmx}
\usepackage{xmpmulti}
\usepackage[T1]{fontenc}
\usepackage[utf8]{inputenc}
\usepackage{cmap}
\usepackage{ifthen}
\usepackage{type1ec}
\usepackage{concrete}
\usepackage{tikz}
\usetikzlibrary{positioning,arrows,decorations.pathreplacing,decorations.text,shapes,calc,fit}
\DeclareFontFamily{T1}{ccr}{}
\DeclareFontShape{T1}{ccr}{m}{n}{<5><6><7><8><9><10>gen*eorm<10->eorm10}{}
\DeclareFontShape{T1}{ccr}{m}{sl}{<5><6><7><8><9><10>gen*eosl<10->eosl10}{}
\DeclareFontShape{T1}{ccr}{m}{it}{<->eoti10}{}
\DeclareFontShape{T1}{ccr}{m}{sc}{<->eocc10}{}
\DeclareFontShape{T1}{ccr}{bx}{n}{<->ssub*cmr/bx/n}{}
\DeclareFontShape{T1}{ccr}{bx}{sl}{<->ssub*cmr/bx/sl}{}
\DeclareFontShape{T1}{ccr}{bx}{it}{<->ssub*cmr/bx/it}{}
\DeclareFontShape{T1}{ccr}{sbc}{n}{<->ssubf*ecssdc10}{}
\usepackage[T1]{fontenc}
\renewcommand{\sfdefault}{\rmdefault}
\renewcommand{\familydefault}{\rmdefault}
\renewcommand{\bfdefault}{m}
\newcommand{\nodewatcher}{\emph{nodewatcher }}
\newcommand{\wlanslovenija}{\emph{wlan slovenija }}

\newcommand{\ifstrempty}[3]{%
\def\reallyempty{}%
\def\ifarg{#1}%
\ifx\ifarg\reallyempty%
{#2}
\else
{#3}
\fi%
}

\newcommand{\wrapframe}[1]{
\begin{frame}
 #1
\end{frame}
}

\newenvironment{changemargin}[2]{%
\begin{list}{}{%
\setlength{\topsep}{0pt}%
\setlength{\leftmargin}{#1}%
\setlength{\rightmargin}{#2}%
\setlength{\listparindent}{\parindent}%
\setlength{\itemindent}{\parindent}%
\setlength{\parsep}{\parskip}%
}%
\item[]}{\end{list}}

\newcommand{\simpleslide}[3]{\wrapframe{
\ifstrempty{#1}{\vbox{ \ } 

}{
{\footnotesize #1}
} \vfill

\begin{center}
#2
\end{center}
\vfill
\ifstrempty{#3}{ \ }{
\begin{flushright}
{\footnotesize #3}
\end{flushright}
}
}}

\newcommand{\simpleslideimage}[3][width]{\wrapframe{
\begin{center}
\ifthenelse{\equal{#1}{width}}{
\includegraphics[width=\pwidth]{#2}}{
\ifthenelse{\equal{#1}{height}}{
\includegraphics[height=\pheight]{#2}}{
\includegraphics{#2}}}
\\ {\footnotesize #3}
\end{center}}
}

\newcommand{\fullslideimage}[1]{
\begin{frame}[plain]
\begin{changemargin}{-1cm}{-1cm}
\begin{center}
\includegraphics[width=\paperwidth,height=\paperheight,keepaspectratio]{#1}
\end{center}
\end{changemargin}
\end{frame}
}


\newcommand{\rightfooter}[1]{\vfill\hfill #1}

\newlength{\pheight}
\newlength{\pwidth}
\setlength{\pheight}{0.6\paperheight}
\setlength{\pwidth}{0.8\paperwidth}

\title{WiFi - Wireless Fidelity 802.11}

%\subtitle {Sežana, 23.6.2012}

%\author{Luka Čehovin}

%\institute{wlan slovenija}

\date[] {1.0}

\begin{document}

% makes a title slide
\maketitle

\simpleslide{kaj}{
\begin{itemize}
\item strojna oprema
\item zakonske omejitve
\item načrtovanje povezav
\item delovanje WiFija in TCP/IP
\end{itemize}
}{}

\simpleslideimage[height]{clipart/wr741nd-4v20-4.jpg}{Router TP-LINK WR741ND}

\simpleslide{WR741ND}{
\begin{itemize}
\item CPU - Atheros AR9331 @ 350 MHz
\item RAM - 32MB
\item Flash - 4MB
\item LAN - 4x 100Mbps Auto MDI/MDIX
\item WAN - 1x 100Mbps Auto MDI/MDIX
\item WiFi - 802.11 b/g/n 150Mbps (130Mbps real) 20dBm - 100mW @ 2.4GHz
\end{itemize}
}{}

\simpleslideimage[height]{clipart/TP-LINK-serial.pdf}{Voltage level shifter}


\simpleslide{Izbira lokacije in antene za \wlanslovenija točko. }{
\begin{enumerate}
\item lokacija z čim večjim vidnim poljem (streha, okenska polica, fasada,...)
\item območje potencialnih uporabnik (naselja, parki, javne površine, hribi...)
\item primerna antena glede na ovire (omni za streho, panel za fasado, yagi za usmerjene povezave)
\item sodelovanje z drugimi uporabniki
\end{enumerate}
}{}


\simpleslide{WiFi zakonodaja.}{
\begin{itemize}
\item omejena izotropsko ekvivalenta izsevana moč - EIRP router(dBm)+antena(dB)
\item 20dBm EIRP @ 2.4GHz
\item 23dBm EIRP @ 5150-5350 MHz indoor
\item 30dBm EIRP @ 5475-5725 MHz
\end{itemize}
}{}

\simpleslide{}{Veliko malih šepetalcev.}{Namesto velikih oddajnih stolpov.}

\tikzstyle{vertex}=[circle,text=blue!75,fill=blue!25,minimum size=20pt,inner sep=0pt]
\tikzstyle{selected vertex} = [vertex, fill=red!24, text=red]
\tikzstyle{edge} = [draw=black,thick,-]
\tikzstyle{weight} = [font=\scriptsize]
\tikzstyle{selected edge} = [draw,line width=5pt,-,red!50]
\tikzstyle{destination} = [vertex,fill=orange!24, text=orange]

\pgfdeclarelayer{background}
\pgfsetlayers{background,main}

\simpleslide{WiFi in komunikacija IP.}{Omrežne plasti.}{Lego kocke.}

\simpleslideimage[height]{clipart/legostack.jpg}{}

\simpleslide{Modularnost.}{Razširljivost.}{Interopabilnost.}

\simpleslide{WiFi. Ethernet.}{Povezavna plast.}{Lokalno (vidno) območje.}

\simpleslide{}{Internetna plast.}{Grafi.}

\simpleslideimage[height]{clipart/network_topology.png}{Izsek grafa topologije \textit{wlan slovenija}}

\simpleslide{Topologije.}{Dostopovne točke vs. mesh omrežje.}{}

\simpleslide{\textit{wlan slovenija}.}{Mesh. Avtonomno omrežje.}{Del prepleta globalnih omrežij, Interneta.}

\simpleslide{Usmerjanje na omrežjih IP.}{Dinamični usmerjevalni protokoli.}{Statično usmerjanje.}

\simpleslide{OLSR.}{Dinamično usmerjanje v WiFi mesh omrežjih.}{Cene povezav, metrike in optimalne poti.}

\simpleslide{}{

\begin{tikzpicture}[scale=1.8, auto,swap]
  \foreach \pos/\name in {{(0,2)/a}, {(2,1)/b}, {(4,1)/c},
                         {(0,0)/d}, {(3,0)/e}, {(2,-1)/f}, {(4,-1)/g}}
    \node[vertex] (\name) at \pos {$\name$};
  
  \node[destination] (g) at (4,-1) {$g$};
  
  % Connect vertices with edges and draw weights
  \foreach \source/ \dest /\weight in {b/a/$1.0$, c/b/$1.0$, d/a/$1.21$, d/b/$2.33$,
                                       e/b/$2.17$, e/c/$2.43$, e/d/$1.87$,
                                       f/d/$7.81$, f/e/$3.20$,
                                       g/e/$2.01$, g/f/$1.12$}
      \path[edge] (\source) -- node[weight] {$\weight$} (\dest);
  
  \foreach \vertex in {a,d,e}
    \path<2-> node[selected vertex] at (\vertex) {$\vertex$};
  
  \begin{pgfonlayer}{background}
    \foreach \source / \dest in {a/d,d/e,e/g}
      \path<2->[selected edge] (\source.center) -- (\dest.center);
  \end{pgfonlayer}
\end{tikzpicture}

}{}

\simpleslide{Transportna plast.}{Razbitje na aplikacije in spet potrditve...}{...ali pa tudi ne.}

\simpleslide{}{Aplikacijski nivo.}{Najbližje uporabniku.}

\end{document}
