\documentclass[]{beamer}
\mode<presentation>
{
% use attribute "dark" for dark theme
% use attribute "contrast" for maximum text contrast
\usetheme[]{wlansi}
\usefonttheme[onlymath]{serif}
\setbeamercovered{transparent}
}  

\usepackage{hyperref}
\usepackage[utf8]{inputenc}
\usepackage{mathptmx}
\usepackage{xmpmulti}
\usepackage[T1]{fontenc}
\usepackage[utf8]{inputenc}
\usepackage{cmap}
\usepackage{ifthen}
\usepackage{type1ec}
\usepackage{concrete}
\DeclareFontFamily{T1}{ccr}{}
\DeclareFontShape{T1}{ccr}{m}{n}{<5><6><7><8><9><10>gen*eorm<10->eorm10}{}
\DeclareFontShape{T1}{ccr}{m}{sl}{<5><6><7><8><9><10>gen*eosl<10->eosl10}{}
\DeclareFontShape{T1}{ccr}{m}{it}{<->eoti10}{}
\DeclareFontShape{T1}{ccr}{m}{sc}{<->eocc10}{}
\DeclareFontShape{T1}{ccr}{bx}{n}{<->ssub*cmr/bx/n}{}
\DeclareFontShape{T1}{ccr}{bx}{sl}{<->ssub*cmr/bx/sl}{}
\DeclareFontShape{T1}{ccr}{bx}{it}{<->ssub*cmr/bx/it}{}
\DeclareFontShape{T1}{ccr}{sbc}{n}{<->ssubf*ecssdc10}{}
\usepackage[T1]{fontenc}
\renewcommand{\sfdefault}{\rmdefault}
\renewcommand{\familydefault}{\rmdefault}
\renewcommand{\bfdefault}{m}
\newcommand{\nodewatcher}{\emph{nodewatcher }}
\newcommand{\wlanslovenija}{\emph{wlan slovenija }}

\newcommand{\ifstrempty}[3]{%
\def\reallyempty{}%
\def\ifarg{#1}%
\ifx\ifarg\reallyempty%
{#2}
\else
{#3}
\fi%
}

\newcommand{\wrapframe}[1]{
\begin{frame}
 #1
\end{frame}
}

\newenvironment{changemargin}[2]{%
\begin{list}{}{%
\setlength{\topsep}{0pt}%
\setlength{\leftmargin}{#1}%
\setlength{\rightmargin}{#2}%
\setlength{\listparindent}{\parindent}%
\setlength{\itemindent}{\parindent}%
\setlength{\parsep}{\parskip}%
}%
\item[]}{\end{list}}

\newcommand{\simpleslide}[3]{\wrapframe{
\ifstrempty{#1}{\vbox{ \ } 

}{
{\footnotesize #1}
} \vfill

\begin{center}
#2
\end{center}
\vfill
\ifstrempty{#3}{ \ }{
\begin{flushright}
{\footnotesize #3}
\end{flushright}
}
}}

\newcommand{\simpleslideimage}[2]{\wrapframe{
\begin{center}
\includegraphics[width=0.8\paperwidth,height=0.6\paperheight,keepaspectratio]{#1}
\\ {\footnotesize #2}
\end{center}}
}

\newcommand{\fullslideimage}[1]{
\begin{frame}[plain]
\begin{changemargin}{-1cm}{-1cm}
\begin{center}
\includegraphics[width=\paperwidth,height=\paperheight,keepaspectratio]{#1}
\end{center}
\end{changemargin}
\end{frame}
}

\newcommand{\rightfooter}[1]{\vfill\hfill #1}

\title{Omrežje \wlanslovenija}

\subtitle{Jernej Kos \\ SINOG 1.5}

%\author{}

%\institute{}

\date[]{4.12.2014}

\begin{document}

% makes a title slide
\maketitle

\pgfdeclarelayer{background}
\pgfsetlayers{background,main}

\simpleslide{}{\wlanslovenija -- odprto brezžično omrežje Slovenije \\ https://wlan-si.net}{https://grow.wlan-si.net \\ https://dev.wlan-si.net}

\simpleslide{}{Pobuda za izgradnjo odprtega brezžičnega \\
omrežja po Sloveniji, temelječega na sodelovanju, \\
izmenjavi znanja in izkušenj ter skupnosti.}{}

\simpleslide{Organska rast omrežne infrastrukture.}{Izzivi.}{Koordinacija skupnosti.}

\simpleslide{Omrežje.}{Mesh omrežje.}{Brezžično mesh omrežje.}

\simpleslide{Omrežje.}{Povezljivost na katerikoli možen način.}{WiFi, Ethernet, L2TPv3, KORUZA...}

\fullslideimage{clipart/shema-podolgovata.png}

\fullslideimage{images/network_topology.png}

\fullslideimage{images/node-map-slovenia.png}

\fullslideimage{images/node-map-haloze.png}

\simpleslide{Omrežje.}{Dinamično usmerjanje, protokoli, metrike.}{OLSR, Babel, Batman.}

\simpleslide{Omrežje.}{Naprave.}{Linux, OpenWrt. \\ http://www.openwrt.org}

\simpleslide{$\sim$ 1000 geografsko razpršenih točk.}{Naprave. Konfiguracija. Postavitev.}{Upravljanje omrežja.}

\simpleslide{Upravljanje omrežja.}{\huge \nodewatcher}{Razširljiva platforma za upravljanje \\ velikih (brezžičnih) mesh omrežij.}

\simpleslideimage{clipart/device-mgmt-cycle.pdf}{Cikel postavitve naprav.}

\simpleslideimage{images/shema.png}{Konfiguracija. Neodvisna od platforme. \\ }

\simpleslide{Firmware.}{Transformacija konfiguracije.}{Generator.}

\simpleslide{}{Postavitev.}{}

\fullslideimage{images/solar-1.jpg}

\fullslideimage{images/solar-2.jpg}

\fullslideimage{images/neubergerjeva.jpg}

\fullslideimage{images/glinska.jpg}

\fullslideimage{images/rozmanova-1.jpg}

\fullslideimage{images/rozmanova-2.jpg}

\fullslideimage{images/urban.jpg}

\fullslideimage{images/linkslatinaurban06.jpg}

\fullslideimage{images/PohorjeTestiranje.png}

\fullslideimage{images/koruza-ijs-teslova-4.jpg}

\simpleslide{Nadzor.}{Podatki.}{Skladnost s konfiguracijo. \\ Samodejno odkrivanje napak v omrežju.}

\simpleslide{Pridobivanje podatkov.}{nodewatcher-agent (JSON), \\ SNMP, \\ ...}{Modularna platforma.}

\simpleslide{Vse le ne gre brez žic.}{{\huge \textit{Tunneldigger}} \\ {...in zakaj ne OpenVPN?}}{L2TPv3 broker in odjemalec.}

\simpleslide{Tunneldigger. Layer 2 pseudo-wire.}{Context switchi, kopiranje. \\ L2TPv3 podpora v Linux jedru.}{NAT, PMTU, ...}

\simpleslide{Kompleksnost postavitve?}{Docker kontejnerji.}{Za vse komponente platforme.}

\subtitle{https://wlan-si.net \\ jernej@kos.mx}
\date[]{}

\maketitle

\end{document}
