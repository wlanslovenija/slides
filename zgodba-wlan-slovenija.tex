\documentclass[]{beamer}
\mode<presentation>
{
% use attribute "dark" for dark theme
% use attribute "contrast" for maximum text contrast
\usetheme[]{wlansi}
\usefonttheme[onlymath]{serif}
\setbeamercovered{transparent}
}  

\usepackage{hyperref}
\usepackage[utf8]{inputenc}
\usepackage{mathptmx}
\usepackage{xmpmulti}
\usepackage[T1]{fontenc}
\usepackage[utf8]{inputenc}
\usepackage{cmap}
\usepackage{ifthen}
\usepackage{type1ec}
\usepackage{concrete}
\DeclareFontFamily{T1}{ccr}{}
\DeclareFontShape{T1}{ccr}{m}{n}{<5><6><7><8><9><10>gen*eorm<10->eorm10}{}
\DeclareFontShape{T1}{ccr}{m}{sl}{<5><6><7><8><9><10>gen*eosl<10->eosl10}{}
\DeclareFontShape{T1}{ccr}{m}{it}{<->eoti10}{}
\DeclareFontShape{T1}{ccr}{m}{sc}{<->eocc10}{}
\DeclareFontShape{T1}{ccr}{bx}{n}{<->ssub*cmr/bx/n}{}
\DeclareFontShape{T1}{ccr}{bx}{sl}{<->ssub*cmr/bx/sl}{}
\DeclareFontShape{T1}{ccr}{bx}{it}{<->ssub*cmr/bx/it}{}
\DeclareFontShape{T1}{ccr}{sbc}{n}{<->ssubf*ecssdc10}{}
\usepackage[T1]{fontenc}
\renewcommand{\sfdefault}{\rmdefault}
\renewcommand{\familydefault}{\rmdefault}
\renewcommand{\bfdefault}{m}
\newcommand{\nodewatcher}{\emph{nodewatcher }}
\newcommand{\wlanslovenija}{\emph{wlan slovenija }}

\newcommand{\ifstrempty}[3]{%
\def\reallyempty{}%
\def\ifarg{#1}%
\ifx\ifarg\reallyempty%
{#2}
\else
{#3}
\fi%
}

\newcommand{\wrapframe}[1]{
\begin{frame}
 #1
\end{frame}
}

\newenvironment{changemargin}[2]{%
\begin{list}{}{%
\setlength{\topsep}{0pt}%
\setlength{\leftmargin}{#1}%
\setlength{\rightmargin}{#2}%
\setlength{\listparindent}{\parindent}%
\setlength{\itemindent}{\parindent}%
\setlength{\parsep}{\parskip}%
}%
\item[]}{\end{list}}

\newcommand{\simpleslide}[3]{\wrapframe{
\ifstrempty{#1}{\vbox{ \ } 

}{
{\footnotesize #1}
} \vfill

\begin{center}
#2
\end{center}
\vfill
\ifstrempty{#3}{ \ }{
\begin{flushright}
{\footnotesize #3}
\end{flushright}
}
}}

\newcommand{\simpleslideimage}[3][width]{\wrapframe{
\begin{center}
\ifthenelse{\equal{#1}{width}}{
\includegraphics[width=\pwidth]{#2}}{
\ifthenelse{\equal{#1}{height}}{
\includegraphics[height=\pheight]{#2}}{
\includegraphics{#2}}}
\\ {\footnotesize #3}
\end{center}}
}

\newcommand{\fullslideimage}[1]{
\begin{frame}[plain]
\begin{changemargin}{-1cm}{-1cm}
\begin{center}
\includegraphics[width=\paperwidth,height=\paperheight,keepaspectratio]{#1}
\end{center}
\end{changemargin}
\end{frame}
}


\newcommand{\rightfooter}[1]{\vfill\hfill #1}

\newlength{\pheight}
\newlength{\pwidth}
\setlength{\pheight}{0.8\paperheight}
\setlength{\pwidth}{0.8\paperwidth}

\title{Zgodba \wlanslovenija}

%\subtitle {Sežana, 23.6.2012}

%\author{Luka Čehovin}

%\institute{wlan slovenija}

\date[] {1.0}

\begin{document}

% makes a title slide
\maketitle


\pgfdeclarelayer{background}
\pgfsetlayers{background,main}

\simpleslide{}{\emph{wlan slovenija} -- odprto brezžično omrežje Slovenije \\ http://wlan-si.net}{http://grow.wlan-si.net \\ http://dev.wlan-si.net}

\simpleslide{Začetek.}{2006\\Predstavitev mesh omrežja FunkFeuer v Kiberpipi.\\Navdušenci pograbijo idejo.\\
~\\
2009\\
\textit{wlan ljubljana} -- odprto brezžično omrežje Ljubljane \\ 28 točk, 13000 ne-unikatnih uporabnikov\\
~\\
2012\\
\wlanslovenija -- odprto brezžično omrežje Slovenije \\
150+ nodes, 178000 ne-unikatnih uporabnikov
}{}


\simpleslideimage[width]{clipart/growth.png}{Rast omrežja.}

\simpleslide{Internetna povezljivost Slovenia.}{
širokopasovni Internet (xDSL ali kableski) -- 60\% gospodinjstev -- '10\\ 
FTTH v večjih mestih\\
76\% populacije uporablja Internet\\
500\% rast dostopnosti do širokopasovnega Interneta '05-'10\\
slaba dostopnost v ruralnih obnočjih
}{}

\simpleslide{Težave.}{Višek širokopasovne kapacitete v mestih.\\
Deljenje povezav za omogočanje dostopa na javnih površinah.\\
Pomanjkanje širokopasovne infrastrukture v ruralnih območjih.\\
Deljenje povezav za osnovni dostop.
}{}

\simpleslide{Neuspeh}{
Občine v Halozah uvajajo spletne portale za občane.\\
Velik del gospodinjstev nima možnosti širokopasovnega dostopa.\\
~\\
Prebivalci vzpostavijo svoje omrežje.\\
}{Uspeh!}

\simpleslideimage[width]{clipart/haloze.png}{Brezžično omrežje v Halozah.}

\simpleslide{Cilji!}{
Preprosto in učinkovito deljenje Interneta.\\
Razvoj preprostih, učinkovitih in odprtokodnih rešitev.\\
Promocija odprtih družbenih omrežij.\\
Deljenje znanja.\\
}{}

\simpleslide{}{\emph{wlan slovenija} -- odprto brezžično omrežje Slovenije \\ http://wlan-si.net}{http://grow.wlan-si.net \\ http://dev.wlan-si.net}

\simpleslide{Omrežje}{Mesh omrežje.}{Brezžično mesh omrežje.}

\simpleslideimage[height]{clipart/slovenija.png}{}
\simpleslideimage[height]{clipart/preplet.png}{}

\simpleslide{Širjenje omrežja:}{Sodelovanje posameznikov.  \\ Deljenje lokacije. \\ Prispevanje denarja. \\ }{Prekomejna povezljivost.}
\simpleslideimage[height]{clipart/PohorjeTestiranje.png}{}
\simpleslide{Rezultati.}{Avtonomno omrežje.  \\ Neomejena povezljivost. \\Brezplačno. \\ }{Nov medij za komunikacije. }
\simpleslideimage[width]{clipart/dilbert.jpg}{}
\simpleslideimage[width]{clipart/awmn1}{AWMN Athenes}
\simpleslideimage[width]{clipart/awmn2}{AWMN Athenes}


\end{document}

\simpleslide{}{}{}
